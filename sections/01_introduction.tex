\section{Pendahuluan}
\label{sec:Pendahuluan}
\setlength{\parskip}{0pt}
% Prevent footnote splitting dan flexible page layout
\interfootnotelinepenalty=10000
\clubpenalty=10000
\widowpenalty=10000
\raggedbottom

% APA 7 footnote command
\newcommand{\abaur}[1]{%
  \footnote{%
    \fontsize{10}{12}\selectfont%
    \setlength{\parindent}{0.5in}%
    \setlength{\parskip}{0pt}%
    \raggedright%
    #1%
  }%
}

\dropcap{Isu} pajak di Indonesia selalu sensitif. Setiap wacana kenaikan tarif—betapapun masuk akal dari sudut pandang ekonomi—acapkali disambut oleh pola yang sama: kekhawatiran publik, penolakan dari berbagai kalangan, dan akhirnya lahir kompromi politik. Episode PPN 12 persen pada akhir 2024 adalah contoh terbaru dari siklus ini. Dalam konteks tersebut, dua artikel Budiawan Sidik A. di \textit{Kompas.id}—“Kunci Pembangunan Ekonomi Bukan Hanya soal Besaran Pajak, melainkan Juga Tata Kelola” (24 Desember 2024) dan “Kebijakan Pajak yang Rasional Menstimulasi Kemajuan Ekonomi Nasional” (2 Januari 2025)—menawarkan perspektif yang relevan. Penempatan waktunya pun menarik: artikel pertama muncul di tengah gejolak opini publik, sedangkan artikel kedua hadir setelah pemerintah mengambil jalan tengah dengan membatasi PPN 12 persen hanya untuk barang mewah.

Sebagai peneliti di Litbang \textit{Kompas}, Budiawan menggunakan pendekatan berbasis data dan argumen yang tertata. Perbandingan lintas negara, analisis kelompok pendapatan, dan diagnosis struktural disusun dengan teliti (dan, dalam banyak hal, meyakinkan). Melalui tulisan-tulisannya, Ia tidak sekadar memperdebatkan soal tarif pajak dan justru mengajak kita melihat diskursus perpajakan dari sudut yang lebih luas dan, barangkali, lebih ``menenangkan'': bahwa masalah utama bukan hanya soal berapa persen tarif yang ditetapkan, melainkan seberapa efektif sistem pemungutannya bekerja. Lewat perbandingan lintas negara, termasuk negara-negara tetangga dan negara maju yang berhasil menaikkan rasio pajak mereka secara signifikan, Budiawan menyampaikan satu hal penting: Indonesia belum berada di posisi yang seharusnya.\catatan{Menariknya, beberapa negara dengan tarif PPN yang bahkan \textit{lebih rendah} justru mencatatkan \textit{tax to GDP ratio} yang jauh lebih tinggi. Artinya, masalah kita bukan pada tarif, tapi pada efektivitas dan kepatuhan—dan mungkin juga soal kepercayaan masyarakat terhadap sistem itu sendiri.} Tulisan-tulisannya, secara implisit, juga membawa semacam dorongan psikologis: bahwa \textit{we’re not doomed}—kita hanya belum optimal. Ketertinggalan ini bukan takdir; ia bisa diperbaiki, \texit{asalkan} kita mau mengubah cara pandang dan cara kelola. Dalam konteks rencana kenaikan tarif PPN menjadi 12 persen, Budiawan tidak serta-merta menolak, tetapi justru menawarkan penjelasan atas resistensi publik yang muncul. Di tengah kebutuhan belanja negara yang memang semakin besar, kekhawatiran masyarakat justru mengarah ke hal-hal yang lebih mendasar: potensi penurunan daya beli, perlambatan ekonomi, dan (ini penting) persepsi bahwa legitimasi sosial atas kebijakan fiskal belum benar-benar terbentuk.

Penelitian \textit{Kompas} yang dibawakan Budiawan, misalnya, menunjukkan bahwa penolakan terhadap kenaikan tarif datang dari hampir semua lapisan sosial—dari kelompok berpenghasilan tinggi hingga masyarakat berpendapatan rendah-dan mereka meminta penundaan pelaksanaan kebijakan hingga kondisi ekonomi membaik.\catatan{Model teoritis Timmons menunjukkan bahwa kepatuhan pajak bergantung pada persepsi warga tentang \textit{quid pro quo}—apa yang mereka terima sebagai imbalan dari pajak yang dibayar. Dalam konteks Indonesia, survei \textit{Kompas} yang dikutip Budiawan, menurut penulis ulasan ini, sebenarnya menyiratkan krisis legitimasi yang dalam: ketika semua lapisan sosial menolak kenaikan tarif, ini bukan sekadar soal ekonomi, tetapi soal kepercayaan. (Atau, dalam bahasa yang lebih kasar: ``Untuk apa saya bayar pajak kalau hasilnya tidak jelas?'')} Tanpa legitimasi yang kuat, kebijakan pajak yang baik di atas kertas pun bisa kehilangan daya dorongnya. Dan di titik ini, tulisan Budiawan tidak hanya menyodorkan kritik, tapi juga memberi harapan—bahwa ada jalan lain yang lebih masuk akal: bukan sekadar menaikkan tarif, tapi memperkuat kepercayaan, membangun sistem, dan memperbaiki hubungan antara negara dan warganya.

Kedua tulisan tersebut, menurut penulis, secara implisit juga berusaha menempatkan diskusi pajak dalam konteks struktural yang lebih luas—menyentuh aspek legitimasi sosial kebijakan fiskal, kapasitas negara dalam membangun kepercayaan publik, dan tantangan membiayai pembangunan tanpa menimbulkan beban berlebih pada masyarakat rentan. Namun, yang justru luput dibicarakan (dan inilah yang tulisan ini ingin gali lebih dalam) adalah kenyataan bahwa reformasi kelembagaan, betapapun esensialnya, tidak datang tanpa ongkos: modernisasi sistem informasi, perbaikan birokrasi (termasuk insentif agar aparatur tak tergoda rente), dan investasi jangka panjang dalam SDM semuanya membutuhkan biaya besar. Maka timbul pertanyaan yang tak mudah dielakkan: jika \textit{tax-to-GDP ratio} kita masih stagnan (dan di banyak kasus bahkan turun sejak pandemi), dari mana sumber daya fiskal untuk semua ini berasal—selain dari \textit{memaksa kenaikan} penerimaan pajak itu sendiri melalui kenaikan tarif umum PPN—yang secara administratif paling mudah dilakukan?.\catatan{yang, ironisnya, justru menjadi titik tolak kegelisahan awal publik} Di sinilah artikel Budiawan tampak konservatif: alih-alih mendorong terobosan fiskal yang ekspansif, ia seolah mengafirmasi \textit{status quo} dengan menekankan optimalisasi tata kelola (yang tentu penting, tapi apakah cukup?).

Dengan latar tersebut, ulasan ini akan dibagi ke dalam tiga bagian utama untuk memudahkan pembahasan. Bagian \textit{\hyperref[sec:Ringkasan]{Ringkasan Kedua Artikel}} akan memaparkan isi pokok kedua artikel Budiawan secara sistematis. Bagian \textit{\hyperref[sec:Analisis & Evaluasi]{Analisis}} akan menelusuri argumen, kerangka logis, serta asumsi yang melandasinya, termasuk implikasi yang mungkin luput dibahas. Terakhir, bagian \textit{\hyperref[sec:Kesimpulan]{Kesimpulan}} akan merangkum temuan utama ulasan ini serta memberikan refleksi terhadap relevansi dan keterbatasan argumen tersebut bagi diskursus perpajakan di Indonesia.
\clearpage

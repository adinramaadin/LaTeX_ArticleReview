\section{Kesimpulan}
\label{sec:Kesimpulan}
Budiawan patut diapresiasi atas kontribusinya yang signifikan dalam memperkaya wacana publik tentang perpajakan. Peta yang ia susun berangkat dari data yang kuat, observasi yang presisi, dan strategi retorika yang cermat. Ia berhasil mengangkat perdebatan dari sekadar persoalan tarif menuju pembahasan tentang tata kelola dan kapasitas administrasi. Pendekatan ini penting, sebab tanpa fondasi birokrasi yang memadai, kebijakan pajak kerap menjadi wacana tanpa daya paksa. Di tengah debat publik yang sering terjebak pada narasi jangka pendek, intervensi Budiawan mengingatkan kita pada pentingnya stabilitas sistem sebagai syarat awal.

Fokus yang kuat pada optimalisasi administrasi, walau bernilai strategis, menyisakan ruang kosong pada horizon kebijakan. Pajak bukan sekadar masalah teknis yang bisa diselesaikan di ruang rapat kementerian; ia adalah produk dari \textit{political settlement}. Keberhasilan reformasi sering kali ditentukan bukan hanya oleh rapi atau tidaknya sistem administrasi, tetapi oleh kesediaan politik untuk membuka jalur kebijakan baru. Dalam hal ini, Budiawan memilih wilayah yang aman: ia menavigasi peta yang sudah ada tanpa mencoba menandai wilayah tak dikenal yang mungkin berisiko namun potensial.

Pendekatan alternatif mengajak kita melihat pajak sebagai bagian dari proyek politik yang melibatkan negosiasi legitimasi antara negara dan warganya. Kenaikan rasio pajak tidak hanya menuntut perbaikan mekanisme pemungutan, tetapi juga memerlukan kontrak sosial yang meyakinkan publik bahwa beban pajak dibalas dengan manfaat yang nyata. Dalam bahasa sederhana, keberhasilan reformasi fiskal tidak diukur dari berapa persen tarif dinaikkan, tetapi dari sejauh mana publik melihat pajak sebagai investasi bersama.

Peta yang digambarkan Budiawan sudah memuat jalur yang terbukti aman. Tantangan berikutnya adalah membayangkan jalur yang lebih berani: strategi yang mungkin lebih kompleks, bahkan spekulatif, namun memiliki potensi untuk memecah kebuntuan lama. Pendekatan seperti ini bukan berarti meninggalkan prinsip kehati-hatian; ia justru bertolak dari asumsi bahwa stabilitas administrasi adalah landasan bagi inovasi kebijakan. Optimalisasi teknis penting, tetapi imajinasi politik tidak kalah mendesak.

Paradoks fiskal Indonesia tidak akan terselesaikan jika kita hanya bergerak lebih cepat di lintasan yang sama. Ia memerlukan lompatan yang menembus batas lama, disertai peta baru yang menggabungkan ketepatan teknokratis dengan keberanian politik. Budiawan telah memulai langkah penting dengan memperjelas rute yang ada; langkah berikutnya adalah memperluas cakrawala itu. Sejarah reformasi menunjukkan bahwa perubahan besar jarang datang dari jalan utama—ia justru muncul dari jalur sempit yang awalnya tak terlihat, tetapi akhirnya mengubah arah perjalanan.

\clearpage
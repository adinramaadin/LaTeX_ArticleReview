\section{Ringkasan Kedua Artikel}
\label{sec:Ringkasan}
\setlength{\parskip}{0pt}
% Prevent footnote splitting dan flexible page layout
\interfootnotelinepenalty=10000
\clubpenalty=10000
\widowpenalty=10000
\raggedbottom

Kedua artikel Budiawan Sidik A. menyajikan analisis komprehensif tentang perpajakan Indonesia melalui pendekatan data empiris dan perbandingan internasional. Bagian ini merangkum secara detail substansi, temuan, dan argumen yang dibangun dalam masing-masing artikel tanpa interpretasi atau evaluasi kritis.

\subsection{Artikel Pertama: Substansi dan Temuan Empiris}

Artikel \textit{``Kunci Pembangunan Ekonomi Bukan Hanya soal Besaran Pajak, melainkan Juga Tata Kelola''} dibuka dengan uraian ihwal hasil jajak pendapat \textit{Kompas} pada awal Desember 2024. Survei melibatkan 625 responden dari 38 provinsi dengan metode acak, memiliki \textit{margin of error} ±3,92 persen pada tingkat kepercayaan 95 persen.

\textbf{Temuan survei menunjukkan penolakan yang relatif seragam lintas lapisan sosial ekonomi:} kelompok ekonomi bawah menolak sebesar 50,6 persen, menengah bawah 29,1 persen, menengah atas 27,9 persen, dan atas 25,8 persen. Yang lebih penting, mayoritas dari semua lapisan itu menghendaki penundaan: 53 persen responden meminta kebijakan ditangguhkan hingga keadaan ekonomi membaik, sementara hanya 11,8 persen mendukung penerapan sesuai jadwal (dan ini menarik—karena jarang sekali ada “kebijakan fiskal” yang bisa menyatukan keresahan dari bawah sampai atas).

Budiawan kemudian menyinggung landasan yuridis, merujuk pada UU No. 7 Tahun 2021 tentang Harmonisasi Peraturan Perpajakan. Undang-undang tersebut menetapkan PPN 11 persen berlaku mulai 1 April 2022, serta PPN 12 persen selambatnya 1 Januari 2025. Pemerintah berpegang pada asas keadilan dan gotong royong: pihak yang mampu diharapkan berkontribusi sesuai undang-undang, sementara kelompok rentan dijanjikan perlindungan. Selain itu, khusus untuk barang kebutuhan pokok, pemerintah menyiapkan pembebasan pajak senilai Rp 265,6 triliun dan menanggung kenaikan 1 persen PPN untuk komoditas strategis seperti tepung terigu, gula industri, dan minyak goreng Minyakita. (ini bagian yang biasanya terdengar manis di podium—tetapi, seperti kita tahu, lapangan punya cara sendiri untuk ``menguji'' janji tersebut).

\textbf{Perbandingan internasional menjadi nadi argumentasi artikel ini.} Budiawan menyajikan data \textit{tax-to-GDP ratio} dan tarif PPN dari sembilan negara text{emerging market dan ASEAN}, sebagaimana tercantum dalam sebagaimana tercantum dalam \textit{Bagan~\ref{tab:international_comparison}}. Di sini Budiawan sedang membangun premis penting: tarif hanyalah wajah luar kebijakan, sedangkan efektivitas sistem terletak pada desain dan kapasitasnya—dan ini, kalau dibaca seksama, adalah kritik terhadap pendekatan \textit{business as usual} dalam fiskal Indonesia.

\begin{table}[hbt!]
\caption{Perbandingan Tarif PPN dan \textit{Tax-to-GDP Ratio} Negara-Negara \textit{Emerging Market} dan ASEAN}
\label{tab:international_comparison}
\centering
\begin{tabular}{lcc}
\textbf{Negara} & \textbf{Tarif PPN/VAT/GST (\%)} & \textbf{\textit{Tax-to-GDP Ratio} (\%)} \\
\midrule
Brasil & 17 & 24,67 \\
India & 18 & 17,33 \\
Turki & 20 & 16,4 \\
Thailand & 7 & 15,14 \\
Filipina & 12 & 14,62 \\
Vietnam & 10 & 12,8 \\
Singapura & 8 & 12,03 \\
Malaysia & 8 & 11,64 \\
\textbf{Indonesia} & \textbf{11} & \textbf{10,4} \\
\end{tabular}
\floatnote{Data tahun 2022 sebagaimana disajikan dalam artikel Budiawan. Indonesia ditampilkan dengan format tebal untuk menekankan posisinya sebagai objek analisis utama.}
\end{table}

Artikel juga menyoroti Penanaman Modal Asing (PMA) sebagai petunjuk perihal daya tarik investasi. Vietnam, misalnya, mencatat PMA 4 persen dari PDB; Singapura jauh lebih tinggi dengan 30 persen; sementara Indonesia hanya 1,61 persen (Dan di sini, Budiawan seakan menyelipkan sindiran halus: tarif pajak bukanlah pangkal tunggal persoalan—lihat saja, Vietnam tidak perlu menurunkan tarifnya serendah mungkin untuk mengundang arus modal). PMA yang tinggi berperan menciptakan lapangan kerja formal, memperluas basis pajak, dan pada akhirnya meningkatkan penerimaan negara. Budiawan menautkan fakta ini dengan mutu tata kelola perpajakan, yang pada gilirannya membentuk iklim usaha dan menentukan daya saing di percaturan global. Dengan kata lain, tarif hanyalah wajah luar kebijakan, sedangkan tata kelola adalah ruhnya—dan di sinilah pekerjaan rumah Indonesia tampak paling berat.

\textbf{Kesimpulan} artikel pertama menegaskan bahwa Vietnam dan Singapura patut dijadikan rujukan—bukan untuk meniru tarifnya semata, tetapi dalam pengelolaan pajak dan kemampuannya menghadirkan investasi asing. Budiawan menyiratkan tesis yang cukup gamblang: persoalan Indonesia bukan kekurangan instrumen fiskal, melainkan keterbatasan kapasitas dan tata kelola.

\subsection{Artikel Kedua: Analisis Pascakeputusan dan Perbandingan Regional}

Artikel \textit{``Kebijakan Pajak yang Rasional Menstimulasi Kemajuan Ekonomi Nasional''} dibuka dengan validasi terhadap langkah pemerintah membatasi PPN 12 persen hanya untuk barang mewah. Budiawan merujuk data BPS 2024 yang menunjukkan masyarakat golongan atas di Indonesia berjumlah 1{,}07 juta orang---sekitar 0{,}38 persen populasi---dengan pengeluaran di atas Rp 9{,}9 juta per bulan. Segmen ini kecil, tetapi strategis; pemerintah agaknya sadar bahwa legitimasi publik lebih mudah dijaga bila kebijakan diarahkan ke kelompok yang dinilai mampu menanggungnya.

\textbf{Analisis struktur pengeluaran menyingkap kontras pola konsumsi yang tajam.} Kelompok atas memiliki porsi pengeluaran untuk kebutuhan pokok (pangan, sandang, papan) yang jauh lebih kecil dibanding kelompok menengah-bawah, sementara belanja untuk hiburan, kendaraan, pendidikan, dan kesehatan---yang kerap masuk kategori mewah---jauh lebih besar. Sebaliknya, kelompok menengah menyumbang 81{,}49 persen dari total pengeluaran nasional pada 2024. Dengan kata lain, jika tujuan fiskal adalah memperluas basis penerimaan, maka penyesuaian kebijakan perlu menyentuh struktur pengeluaran riil masyarakat, bukan sekadar menyesuaikan tarif yang tampak efektif di atas kertas.

Artikel ini kemudian melebarkan cakrawala analisis ke tingkat regional, membandingkan data PDB, pertumbuhan ekonomi, pendapatan per kapita, dan tarif PPN di negara-negara ASEAN. Indonesia, dengan PDB USD 1.371{,}17 miliar, memang ekonomi terbesar ASEAN. Namun, dengan tarif PPN 11 persen dan \textit{tax ratio} 10{,}31 persen, posisinya justru berada di peringkat terbawah dalam efektivitas fiskal di kawasan. Paradoks ini kembali hadir: besar di ukuran, namun tertatih dalam pengelolaan; ibarat raksasa yang belum piawai mengatur langkahnya sendiri.

\textbf{Diagnosis masalah struktural dipaparkan dengan terang.} Budiawan menandai tiga simpul persoalan utama:
\begin{enumerate}
    \item Ketidakakuratan pendataan barang dan jasa di pasar domestik,
    \item Pelaporan data fiktif oleh pelaku usaha untuk mengurangi kewajiban pajak, 
    \item Praktik kolusi antara oknum perpajakan dengan wajib pajak ``nakal''.
\end{enumerate}
Ia menekankan perlunya penegakan hukum yang tegas dan pembenahan tata kelola perpajakan yang bersih. Isu ini terdengar akrab, nyaris menjadi refrein lama; namun pengulangannya memberi sinyal bahwa masalahnya bukan teknis belaka, melainkan struktural---tak cukup diselesaikan dengan tambalan kecil di permukaan.

Analisis Budiawan juga menelaah pendorong utama perekonomian ASEAN: konsumsi rumah tangga (50--70 persen PDB di mayoritas negara), investasi atau pembentukan modal tetap bruto (20--30 persen), dan belanja pemerintah (10--16 persen). Indonesia mencatat belanja pemerintah yang relatif rendah (7{,}45 persen), jauh di bawah Malaysia (11{,}95 persen), Thailand (16{,}64 persen), dan Singapura (10{,}23 persen). Rendahnya belanja ini, dibaca bersama \textit{tax ratio} yang kecil, menunjukkan bahwa negara belum memanfaatkan ruang fiskal untuk memacu pertumbuhan---padahal, ruang itu justru krusial untuk memperkuat basis penerimaan di masa depan.

\textbf{Data FDI \textit{inflows} dan \textit{outflows} menyingkap lanskap investasi regional.} Singapura memimpin dengan FDI \textit{inflow} 34{,}95 persen PDB dan \textit{outflow} 12{,}56 persen, menegaskan perannya sebagai \textit{hub} keuangan kawasan. Vietnam konsisten menarik FDI 4--5 persen PDB. Sementara itu, Indonesia stagnan di \textit{inflow} 1{,}61 persen dan \textit{outflow} minimal 0{,}52 persen. Angka-angka ini menegaskan posisi Indonesia: besar secara ekonomi, namun belum menjadi magnet investasi regional. Dan di era arus modal yang lincah, posisi ini bukan sekadar statistik, melainkan peringatan.

Artikel ini---sebagaimana artikel pertama---ditutup dengan rekomendasi yang menitikberatkan pada tiga kata kunci: transparansi, akuntabilitas, dan efektivitas penggunaan hasil pajak. Budiawan menegaskan, legitimasi sosial kebijakan fiskal hanya akan tegak bila masyarakat menyaksikan hasil konkret dari pajak yang dibayarkan. Dalam hal ini, Vietnam dan Singapura kembali dihadirkan sebagai rujukan---bukan semata pada level teknis, tetapi pada integrasi antara kebijakan fiskal, tata kelola, dan strategi pembangunan ekonomi. Seolah Budiawan ingin berbisik: kita tidak kekurangan instrumen; yang kita perlukan adalah kemauan dan kapasitas untuk memanfaatkannya dengan sungguh-sungguh.
